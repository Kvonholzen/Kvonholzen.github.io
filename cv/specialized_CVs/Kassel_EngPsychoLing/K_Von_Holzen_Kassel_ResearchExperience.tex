%% Template for a CV
%% Author: Rob J Hyndman

\documentclass[10pt,a4paper,]{article}
\usepackage[scaled=0.86]{DejaVuSansMono}
%\usepackage[sfdefault,lf,t]{carlito}

% Change color to blue
\usepackage{color,xcolor}
\definecolor{headcolor}{HTML}{990000}

\usepackage{ifxetex,ifluatex}
\usepackage{fixltx2e} % provides \textsubscript
\ifnum 0\ifxetex 1\fi\ifluatex 1\fi=0 % if pdftex
  \usepackage[T1]{fontenc}
  \usepackage[utf8]{inputenc}
\else % if luatex or xelatex
  \ifxetex
    \usepackage{mathspec}
  \else
    \usepackage{fontspec}
  \fi
  \defaultfontfeatures{Ligatures=TeX,Scale=MatchLowercase}
\fi

\usepackage[utf8]{inputenc}
\usepackage[T1]{fontenc}

% use upquote if available, for straight quotes in verbatim environments
\IfFileExists{upquote.sty}{\usepackage{upquote}}{}
% use microtype if available
\IfFileExists{microtype.sty}{%
\usepackage[]{microtype}
\UseMicrotypeSet[protrusion]{basicmath} % disable protrusion for tt fonts
}{}
\PassOptionsToPackage{hyphens}{url} % url is loaded by hyperref
\usepackage[unicode=true,hidelinks]{hyperref}
\urlstyle{same}  % don't use monospace font for urls
\usepackage{geometry}
\geometry{left=1.75cm,right=1.75cm,top=2.2cm,bottom=2cm}

\usepackage{longtable,booktabs}
% Fix footnotes in tables (requires footnote package)
\IfFileExists{footnote.sty}{\usepackage{footnote}\makesavenoteenv{long table}}{}
\IfFileExists{parskip.sty}{%
\usepackage{parskip}
}{% else
\setlength{\parindent}{0pt}
\setlength{\parskip}{6pt plus 2pt minus 1pt}
}
\setlength{\emergencystretch}{3em}  % prevent overfull lines
\providecommand{\tightlist}{%
  \setlength{\itemsep}{0pt}\setlength{\parskip}{0pt}}
\setcounter{secnumdepth}{0}

% set default figure placement to htbp
\makeatletter
\def\fps@figure{htbp}
\makeatother



\date{September 2023}


\usepackage{paralist,ragged2e,datetime}
\usepackage[hyphens]{url}
\usepackage{fancyhdr,enumitem,pifont}
\usepackage[compact,small,sf,bf]{titlesec}

\RaggedRight
\sloppy

% Header and footer
\pagestyle{fancy}
\makeatletter
\lhead{\sf\textcolor[gray]{0.4}{Research Experience: \@name}}
\rhead{\sf\textcolor[gray]{0.4}{\thepage}}
\cfoot{}
\def\headrule{{\color[gray]{0.4}\hrule\@height\headrulewidth\@width\headwidth \vskip-\headrulewidth}}
\makeatother

% Header box
\usepackage{tabularx}

\makeatletter
\def\name#1{\def\@name{#1}}
\def\info#1{\def\@info{#1}}
\makeatother
\newcommand{\shadebox}[3][.9]{\fcolorbox[gray]{0}{#1}{\parbox{#2}{#3}}}

\usepackage{calc}
\newlength{\headerboxwidth}
\setlength{\headerboxwidth}{\textwidth}
%\addtolength{\headerboxwidth}{0.2cm}
\makeatletter
\def\maketitle{
\thispagestyle{plain}
\vspace*{-1.4cm}
\shadebox[0.9]{\headerboxwidth}{\sf\color{headcolor}\hfil
\hbox to 0.98\textwidth{\begin{tabular}{l}
\\[-0.3cm]
\LARGE\textbf{\@name}
\\[0.1cm]\large Lecturer and Researcher\\[0.6cm]
\normalsize\textbf{Research Experience}\\
\normalsize September 2023
\end{tabular}
\hfill\hbox{\fontsize{9}{12}\sf
\begin{tabular}{@{}rl@{}}
\@info
\end{tabular}}}\hfil
}
\vspace*{0.2cm}}
\makeatother

% Section headings
\titlelabel{}
\titlespacing{\section}{0pt}{1.5ex}{0.5ex}
\titleformat*{\section}{\color{headcolor}\large\sf\bfseries}
\titleformat*{\subsection}{\color{headcolor}\sf\bfseries}
\titlespacing{\subsection}{0pt}{1ex}{0.5ex}

% Miscellaneous dimensions
\setlength{\parskip}{0ex}
\setlength{\parindent}{0em}
\setlength{\headheight}{15pt}
\setlength{\tabcolsep}{0.15cm}
\clubpenalty = 10000
\widowpenalty = 10000
\setlist{itemsep=1pt}
\setdescription{labelwidth=1.2cm,leftmargin=1.5cm,labelindent=1.5cm,font=\rm}

% Make nicer bullets
\renewcommand{\labelitemi}{\ding{228}}

\usepackage{booktabs,fontawesome5}
%\usepackage[t1,scale=0.86]{sourcecodepro}

\name{Katie Von Holzen}
\def\imagetop#1{\vtop{\null\hbox{#1}}}
\info{%
\raisebox{-0.05cm}{\imagetop{\faIcon{map-marker-alt}}} &  \imagetop{\begin{tabular}{@{}l@{}}Hagenring
22, 38106 Braunschweig, Germany\end{tabular}}\\ %
\faIcon{home} & \href{http://Kvonholzen.github.io}{Kvonholzen.github.io}\\% %
%
%
\faIcon{envelope} & \href{mailto:katie.m.vonholzen@gmail.com}{\nolinkurl{katie.m.vonholzen@gmail.com}}\\%
\faIcon{twitter} & \href{https://twitter.com/KatieVonHolzen}{@KatieVonHolzen}\\%
\faIcon{github} & \href{https://github.com/Kvonholzen}{Kvonholzen}\\%
%
%
%
%
}


%\usepackage{inconsolata}


\setlength\LTleft{0pt}
\setlength\LTright{0pt}

% Pandoc CSL macros
\newlength{\cslhangindent}
\setlength{\cslhangindent}{1.5em}
\newlength{\csllabelwidth}
\setlength{\csllabelwidth}{3em}
\newenvironment{CSLReferences}[3] % #1 hanging-ident, #2 entry spacing
 {% don't indent paragraphs
  \setlength{\parindent}{0pt}
  % turn on hanging indent if param 1 is 1
  \ifodd #1 \everypar{\setlength{\hangindent}{\cslhangindent}}\ignorespaces\fi
  % set entry spacing
  \ifnum #2 > 0
  \setlength{\parskip}{#2\baselineskip}
  \fi
 }%
 {}
\usepackage{calc}
\newcommand{\CSLBlock}[1]{#1\hfill\break}
\newcommand{\CSLLeftMargin}[1]{\parbox[t]{\csllabelwidth}{\hfill #1~}}
\newcommand{\CSLRightInline}[1]{\parbox[t]{\linewidth - \cslhangindent - \csllabelwidth}{#1}\vspace{0.8ex}}
\newcommand{\CSLIndent}[1]{\hspace{\cslhangindent}#1}


\def\endfirstpage{\newpage}

\begin{document}
\maketitle


\hypertarget{research-program}{%
\section{Research Program}\label{research-program}}

\hypertarget{program-i-how-do-learners-break-into-word-segmentation}{%
\subsection{Program I: How do learners break into word
segmentation?}\label{program-i-how-do-learners-break-into-word-segmentation}}

In the context of globalization, language competence is key for
effective communication and cultural understanding. As a result, great
importance is given to language instruction, for example in the
instruction of foreign languages such as English, which is considered
the \emph{lingua franca} in many EU member states. Although policy
decisions about the starting age of FL instruction seek to maximize
student-learning outcomes, these decisions are often based upon research
done with adult language learners in immersion contexts or young foreign
language learners with several years of experience. Although certainly
useful, a focus on later outcomes is often blurred by effects of later
instruction, and cannot clarify the underlying mechanisms that drive
different developmental outcomes. \linebreak

This research program examines the early steps to process and learn from
foreign language by studying both young, early L2 learners as well as
more experienced L2 learners. It is divided into 3 themes: 1) The role
of speech modifications for foreign language learners; 2) Phonotactic
cues in foreign speech segmentation; and 3) Top-down lexical knowledge
in foreign speech segmentation. These themes focus on one of the first
tasks facing the language learner: identifying words in the incoming
speech signal, known as word segmentation. Yet, this first task presents
a challenge to the learner, as speech does not contain any ostensive
pauses between words which would signal a word boundary. My goal is to
clarify the capacities and constraints of FL and L2 learning that adult
and child learners possess, especially in regards to word segmentation,
which lay the foundations for long-term achievement.

\hypertarget{program-ii-building-equity-in-develpmental-science}{%
\subsection{Program II: Building equity in develpmental
science}\label{program-ii-building-equity-in-develpmental-science}}

Research on human behavior is often conducted by WEIRD (Western,
educated, industrialized, rich, democratic) researchers studying WEIRD
participants, who only make up 12\% of the global population. The
resulting questions asked as well as the theories developed therefore
don't necessarily apply to the majority of the world's population.
Initiatives such as ManyBabies, a community of members focused on
replication and best practices in developmental science, as well as the
flexibility of testing psycholinguistic research remotely in
participants' own homes have the potential to reach more diverse
populations as well as provide scientists around the world with the
tools they need to conduct robust developmental research. \linebreak

This research program is centered around the ManyBabies-AtHome (MBAH)
project, which is a consortium of developmental researchers working
together to produce a resource-friendly, open-source and accessible
approach to make it possible for online studies live up to their promise
of increasing diversity. We have four aims: 1) Provide tools for online
data collection and processing, 2) Realize global accessibility of
remote testing platforms beyond English-speaking areas, 3) Establish
open protocols for ethical data collection, sharing, and protection, and
4) Create a research network to study diverse populations. I am a member
of the leadership committee of the MBAH project, as well as the project
manager of the Looking-While-Listening sub-project of MBAH. The
Looking-While Listening sub-project applies the aims of the full MBAH
project in their examination of early infant word recognition using
online-methods, in addition to the goal of creating a multilingual and
multicultural study with comparable stimuli across many different
language contexts (10+ languages).

\hypertarget{research-areas}{%
\section{Research Areas}\label{research-areas}}

\begin{longtable}{@{\extracolsep{\fill}}ll}
Phonology & \parbox[t]{0.85\textwidth}{%
\textbf{Phonotactics; Consonant-vowel asymmetry; Cross-linguistic similarity}\hfill{\footnotesize }\newline
  \empty%
  \empty%
\vspace{\parsep}}\\
Lexical Processing & \parbox[t]{0.85\textwidth}{%
\textbf{Phono-lexical interface; Word recognition; Word segmentation}\hfill{\footnotesize }\newline
  \empty%
  \empty%
\vspace{\parsep}}\\
Language acquisition & \parbox[t]{0.85\textwidth}{%
\textbf{Second-language; Foreign-language; Bilingual; First-language}\hfill{\footnotesize }\newline
  \empty%
  \empty%
\vspace{\parsep}}\\
Methodology & \parbox[t]{0.85\textwidth}{%
\textbf{Event-Related Potentials (ERPs); Eyetracking (gaze); Time-course analyses; Longitudinal studies}\hfill{\footnotesize }\newline
  \empty%
  \empty%
\vspace{\parsep}}\\
Open Scholarship & \parbox[t]{0.85\textwidth}{%
\textbf{Large-scale collaborations (Many Babies); Accessibility and diversity; Open and reproducible approaches}\hfill{\footnotesize }\newline
  \empty%
  \empty%
\vspace{\parsep}}\\
\end{longtable}

\pagebreak

\hypertarget{invited-talks-and-conference-activity}{%
\section{Invited Talks and Conference
Activity}\label{invited-talks-and-conference-activity}}

\hypertarget{invited-talks}{%
\subsection{Invited Talks}\label{invited-talks}}

\hypertarget{bibliography}{}
\leavevmode\vadjust pre{\hypertarget{ref-iVH2023a}{}}%
\CSLLeftMargin{1. }%
\CSLRightInline{Von Holzen, K. (2023). \emph{Studying speech
segmentation online using behavioral, eye-tracking, and
electrophysiological methods}. Presented at the Ringvorlesung {``Recent
Advances in Linguistic Methods,''} Ludwig Maximilian University of
Munich, Germany.}

\leavevmode\vadjust pre{\hypertarget{ref-iVH2022d}{}}%
\CSLLeftMargin{2. }%
\CSLRightInline{Von Holzen, K. (2022). \emph{Doyouunderstandwhatimsaying
: How learners break into speech stream segmentation}. Presented at the
HaBilNet2 Workshop, Frankfurt, Germany.}

\leavevmode\vadjust pre{\hypertarget{ref-iVH2022c}{}}%
\CSLLeftMargin{3. }%
\CSLRightInline{Von Holzen, K. (2022). \emph{Using cluster-based
permutation tests to analyse eye-tracking data}. Presented at the CLaS
Eye-tracking Workshop 2, Macquarie University, Australia.}

\leavevmode\vadjust pre{\hypertarget{ref-iVH2022b}{}}%
\CSLLeftMargin{4. }%
\CSLRightInline{Von Holzen, K. (2022).
\emph{DoyouunderstandwhatIamsaying? How learners break into speech
segmentation}. Presented at the LinguisTisch Seminar Series,
Sociolinguistics Lab, Universität Duisburg-Essen, Germany.}

\leavevmode\vadjust pre{\hypertarget{ref-iVH2022a}{}}%
\CSLLeftMargin{5. }%
\CSLRightInline{Von Holzen, K. (2022).
\emph{DoyouunderstandwhatIamsaying? How learners break into speech
segmentation}. Presented at the Linguistics Colloquium, TU Braunschweig,
Germany.}

\leavevmode\vadjust pre{\hypertarget{ref-iVH2021}{}}%
\CSLLeftMargin{6. }%
\CSLRightInline{Von Holzen, K. (2021). \emph{Babies know words, even
when they're mispronounced: An introduction to meta-analyses}. Presented
at the German Seminar Colloquium, Albert-Ludwigs Universitaet Freiburg,
Germany.}

\leavevmode\vadjust pre{\hypertarget{ref-iVH2020}{}}%
\CSLLeftMargin{7. }%
\CSLRightInline{Von Holzen, K. (2020). \emph{Babies know words, even
when they're mispronounced: A meta-analysis of mispronunciation
sensitivity studies}. Presented at the Team Meeting of the Language and
Cognition Group, The Institute of Neuroscience and Cognition, Paris
Descartes Universite, Paris, France.}

\leavevmode\vadjust pre{\hypertarget{ref-iVH2019b}{}}%
\CSLLeftMargin{8. }%
\CSLRightInline{Von Holzen, K. (2019). \emph{On the road to
bilingualism: Foreign speech processing at first exposure in early
childhood}. Presented at the Psychology Institute Colloquium, Johannes
Gutenberg Universitaet Mainz, Germany.}

\leavevmode\vadjust pre{\hypertarget{ref-iVH2019a}{}}%
\CSLLeftMargin{9. }%
\CSLRightInline{Von Holzen, K. (2019). \emph{Babies know words, even
when they're mispronounced: A meta-analysis of mispronunciation
sensitivity}. Presented at the Linguistics Department, City University
of New York, USA.}

\leavevmode\vadjust pre{\hypertarget{ref-iVH2018}{}}%
\CSLLeftMargin{10. }%
\CSLRightInline{Von Holzen, K. (2018). \emph{On the road to
bilingualism: The role of age in foreign speech processing}. Presented
at the Team Meeting of the Language and Cognition Group, The Institute
of Neuroscience and Cognition, Paris Descartes Universite, Paris,
France.}

\leavevmode\vadjust pre{\hypertarget{ref-iVH2016b}{}}%
\CSLLeftMargin{11. }%
\CSLRightInline{Von Holzen, K. (2016). \emph{Phonological and lexical
processing in first language and multilingual acquisition}. Presented at
the Linguistics Faculty, Potsdam, Germany.}

\leavevmode\vadjust pre{\hypertarget{ref-iVH2016a}{}}%
\CSLLeftMargin{12. }%
\CSLRightInline{Von Holzen, K., \& Nazzi, T. (2016). \emph{Origine de
l'asymetrie consonne/voyelle lors du traitement lexical}. Presented at
the Agence nationale de la recherche meeting of the grant funding
programme BLANC 2013, Paris, France.}

\leavevmode\vadjust pre{\hypertarget{ref-iVH2014}{}}%
\CSLLeftMargin{13. }%
\CSLRightInline{Von Holzen, K. (2014). \emph{Segmentation of infant- and
adult-directed speech in 12-month-olds: An ERP study}. Presented at the
Concordia Infant Research Laboratory Team Meeting, Montreal, Canada.}

\leavevmode\vadjust pre{\hypertarget{ref-iVH2012a}{}}%
\CSLLeftMargin{14. }%
\CSLRightInline{Von Holzen, K. (2012). \emph{The cognate facilitation
effect in bilingual and monolingual toddlers}. Presented at the Workshop
on Monolingual and Bilingual Word Recognition and Learning in Infants
and Adults, Basque Center on Cognition, Brain, and Language, San
Sebastian, Spain.}

\leavevmode\vadjust pre{\hypertarget{ref-iVH2012b}{}}%
\CSLLeftMargin{15. }%
\CSLRightInline{Von Holzen, K. (2012). \emph{Learning phonemes from
faces: The role of speaker identity in non-native phoneme
discrimination}. Presented at the Workshop on Monolingual and Bilingual
Word Recognition and Learning in Infants and Adults, Basque Center on
Cognition, Brain, and Language, San Sebastian, Spain.}

\leavevmode\vadjust pre{\hypertarget{ref-iVH2012c}{}}%
\CSLLeftMargin{16. }%
\CSLRightInline{Von Holzen, K. (2012). \emph{Mechanisms underlying
lexicalprocessing in monolingual and bilingual toddlers}. In L. Polka
(Chair), Comparing monolingual and bilingual language acquisition during
infancy. Invited symposium conducted at the XVIIIth Biennial
International Conference on Infant Studies, Minneapolis, MN.}

\hypertarget{conference-talks}{%
\subsection{Conference Talks}\label{conference-talks}}

\hypertarget{bibliography}{}
\leavevmode\vadjust pre{\hypertarget{ref-tW2023}{}}%
\CSLLeftMargin{1. }%
\CSLRightInline{Wulfert, S., Von Holzen, K., Schnieders, M., \& Hopp, H.
(2023). \emph{Use of L1 phonotactics in initial foreign-language speech
segmentation}. Presented at EuroSLA 32, Birmingham, UK.}

\leavevmode\vadjust pre{\hypertarget{ref-tV2022a}{}}%
\CSLLeftMargin{2. }%
\CSLRightInline{Von Holzen, K., Bergmann, C., \&
ManyBabies-AtHome-Consortium. (2022). \emph{ManyBabies-AtHome looking
while listening: Constructing an online, cross-linguistic investigation
of word recognition}. Presented at the 52nd Congress of the German
Psychological Society, Hildesheim, Germany.}

\leavevmode\vadjust pre{\hypertarget{ref-tV2022b}{}}%
\CSLLeftMargin{3. }%
\CSLRightInline{Von Holzen, K., \& ManyBabies-AtHome-Consortium. (2022).
\emph{ManyBabies-AtHome}. Presented at the 2022 Big Team Science
Conference, Virtual Conference.}

\leavevmode\vadjust pre{\hypertarget{ref-tS2021}{}}%
\CSLLeftMargin{4. }%
\CSLRightInline{Schlage, F., \& Von Holzen, K. (2021). \emph{From boomer
to digital native: The influence of anglicisms in german social media
language on language processing}. Presented at the 2021 Conference
Architectures and Mechanisms for Language Processing (AMLaP), Paris,
France.}

\leavevmode\vadjust pre{\hypertarget{ref-tB2021}{}}%
\CSLLeftMargin{5. }%
\CSLRightInline{Boveleth, J., \& Von Holzen, K. (2021). \emph{The
difference in the vocabulary size of children with und without down
syndrome: A meta analysis}. Presented at the Muenster Conference
Linguistic Representtions and Language Processing, Muenster, Germany.}

\leavevmode\vadjust pre{\hypertarget{ref-tVH2021}{}}%
\CSLLeftMargin{6. }%
\CSLRightInline{Von Holzen, K., Harnischmacher, V., \& Schuster, N.
(2021). \emph{Phonetic cues in L2 speech segmentation}. Presented at the
2021 Conference Architectures and Mechanisms for Language Processing
(AMLaP), Paris, France.}

\leavevmode\vadjust pre{\hypertarget{ref-tVH2019}{}}%
\CSLLeftMargin{7. }%
\CSLRightInline{Von Holzen, K., \& Bergmann, C. (2019). \emph{Can a tog
be a dog? A meta-analysis of experimental factors influencing infants'
mispronunciation sensitivity}. In Gonzalez-Gomez, N. (Chair), Big ideas
to tackle small samples. Presented at the Biennial Meeting of the
Society for Research in Child Development, Baltimore, USA.}

\leavevmode\vadjust pre{\hypertarget{ref-tVH2018b}{}}%
\CSLLeftMargin{8. }%
\CSLRightInline{Von Holzen, K., \& Bergmann, C. (2018). \emph{A
meta-analysis of infants' mispronunciation sensitivity development}.
Presented at the 40th Annual Meeting of the Cognitive Science Society,
Madison, WI, USA.}

\leavevmode\vadjust pre{\hypertarget{ref-tVH2018a}{}}%
\CSLLeftMargin{9. }%
\CSLRightInline{Von Holzen, K., \& Newman, R. (2018). \emph{On the road
to bilingualism: The role of native language knowledge in foreign speech
processing}. In Koostra, G. J. and Bosma, E. (Chairs), Cross-language
activation in bilingual children. Presented at the Conference on
Multilingualism, Ghent, Belgium.}

\leavevmode\vadjust pre{\hypertarget{ref-tVH2017}{}}%
\CSLLeftMargin{10. }%
\CSLRightInline{Von Holzen, K., Nishibayashi, L.-L., \& Nazzi, T.
(2017). \emph{An ERP study of consonant and vowel processing of newly
segmented word forms.} Presented at the Workshop on Infant Language
Development, Bilbao, Spain.}

\leavevmode\vadjust pre{\hypertarget{ref-tVH2015}{}}%
\CSLLeftMargin{11. }%
\CSLRightInline{Von Holzen, K., \& Nazzi, T. (2015). \emph{The role of
consonants and vowels in 5- and 8-month-old own name recognition:
Implications for lexical development}. Presented at the second Workshop
on Infant Language Development, 2015, Stockholm, Sweden.}

\leavevmode\vadjust pre{\hypertarget{ref-tVH2014}{}}%
\CSLLeftMargin{12. }%
\CSLRightInline{Von Holzen, K., Wolff, D., \& Mani, N. (2014).
\emph{Segmentation of infant- and adult-directed speech in
12-month-olds: An ERP study}. In K. Von Holzen (Chair), Segmenting words
from continuous speech: Examining varied input sources for infants.
Symposium conducted at the XIXth Biennial International Conference on
Infant Studies, Berlin, Germany.}

\leavevmode\vadjust pre{\hypertarget{ref-tVH2013}{}}%
\CSLLeftMargin{13. }%
\CSLRightInline{Von Holzen, K., Wolff, D., \& Mani, N. (2013).
\emph{Segmentierungsfähigkeit von infant- und adult-directed speech bei
12 monate alten kindern (segmentation of infant- and adult-directed
speech in 12-month-old infants)}. Presented at the 22nd Deutsches EEG/EP
Mapping Meeting, Giessen, Germany.}

\leavevmode\vadjust pre{\hypertarget{ref-tVH2012a}{}}%
\CSLLeftMargin{14. }%
\CSLRightInline{Von Holzen, K., \& Mani, N. (2012). \emph{Learning
phonemes from faces: The role of speaker identity in non-native phoneme
discrimination}. In K. Von Holzen (Chair), Infant phonetic learning in
context: The influence of faces, objects, and words. Symposium conducted
at the XVIIIth Biennial International Conference on Infant Studies,
Minneapolis, MN.}

\leavevmode\vadjust pre{\hypertarget{ref-tVH2012b}{}}%
\CSLLeftMargin{15. }%
\CSLRightInline{Von Holzen, K., \& Mani, N. (2012). \emph{The cognate
facilitation effect in bilingual toddlers}. Presented at the
International Workshop Bilingual and Multilingual Interaction, Bangor,
Wales.}

\leavevmode\vadjust pre{\hypertarget{ref-tVH2012c}{}}%
\CSLLeftMargin{16. }%
\CSLRightInline{Von Holzen, K., \& Mani, N. (2012). \emph{Language
non-selective lexical access in bilingual toddlers}. Presented at the
International Workshop Bilingual and Multilingual Interaction, Bangor,
Wales.}

\leavevmode\vadjust pre{\hypertarget{ref-tVH2011}{}}%
\CSLLeftMargin{17. }%
\CSLRightInline{Von Holzen, K., \& Mani, N. (2011). \emph{Bilingual
phonological priming: An ERP study investigating interconnectivity of
activation in the bilinguals two lexicons at different points in
development.} In J. Mayor (Chair), The emergence of lexical networks in
the second year of life. Symposium conducted at the 17th Meeting of the
European Society for Cognitive Psychology, San Sebastian, Spain.}

\hypertarget{conference-posters}{%
\subsection{Conference Posters}\label{conference-posters}}

\hypertarget{bibliography}{}
\leavevmode\vadjust pre{\hypertarget{ref-pVH2023b}{}}%
\CSLLeftMargin{1. }%
\CSLRightInline{Von Holzen, K., Schnieders, M., \& Hopp, H. (2023).
\emph{Use of L1 lexical overlap in initial foreign-language speech
segmentation}. Presented at the 2023 Conference Architectures and
Mechanisms for Language Processing (AMLaP), San Sebastian, Spain.
\url{https://doi.org/10.6084/m9.figshare.24041904}}

\leavevmode\vadjust pre{\hypertarget{ref-pVH2023d}{}}%
\CSLLeftMargin{2. }%
\CSLRightInline{Von Holzen, K., Schnieders, M., Wulfert, S., \& Hopp, H.
(2023). \emph{Foreign-language speech segmentation in ab initio child
learners: The roles of sublexical and lexical L2 overlap and
phonological awareness}. Accepted for presentation at the 2023 Boston
University Conference on Language Development, Boston, MA.}

\leavevmode\vadjust pre{\hypertarget{ref-pVH2023a}{}}%
\CSLLeftMargin{3. }%
\CSLRightInline{Von Holzen, K., Schnieders, M., Wulfert, S., \& Hopp, H.
(2023). \emph{Lexical overlap in foreign language speech segmentation in
primary-level students}. Presented at EuroSLA 32, Birmingham, UK.
\url{https://doi.org/10.6084/m9.figshare.24041883}}

\leavevmode\vadjust pre{\hypertarget{ref-pVH2023c}{}}%
\CSLLeftMargin{4. }%
\CSLRightInline{Von Holzen, K., Wulfert, S., \& Hopp, H. (2023).
\emph{Use of L1 phonotactics in initial foreign-language speech
segmentation}. Presented at the 2023 Conference Architectures and
Mechanisms for Language Processing (AMLaP), San Sebastian, Spain.
\url{https://doi.org/10.6084/m9.figshare.24041928}}

\leavevmode\vadjust pre{\hypertarget{ref-pF2023a}{}}%
\CSLLeftMargin{5. }%
\CSLRightInline{Flohr, M.-C., Von Holzen, K., \& Schimke, S. (2023).
\emph{Exploring the boundaries of statistical learning: Word
segmentation in a natural language}. Presented at Psycholinguistics in
Flanders (PiF) 2023, Ghent, Belgium.}

\leavevmode\vadjust pre{\hypertarget{ref-pF2023b}{}}%
\CSLLeftMargin{6. }%
\CSLRightInline{Flohr, M.-C., Von Holzen, K., \& Schimke, S. (2023).
\emph{Exploring the boundaries of statistical learning: Word
segmentation in a natural language}. Presented at the 2023 Conference
Architectures and Mechanisms for Language Processing (AMLaP), San
Sebastian, Spain.}

\leavevmode\vadjust pre{\hypertarget{ref-pVH2022b}{}}%
\CSLLeftMargin{7. }%
\CSLRightInline{Von Holzen, K. (2022). \emph{The role of speech
modifications in ab inito learners' initial speech segmentation}. 4th
International Symposium on Bilingual and L2 Processing in Adults and
Children, Tromso, Norway.}

\leavevmode\vadjust pre{\hypertarget{ref-pVH2022a}{}}%
\CSLLeftMargin{8. }%
\CSLRightInline{Von Holzen, K., Bergmann, C., \&
ManyBabies-AtHome-Consortium. (2022). \emph{ManyBabies-AtHome looking
while listening: Constructing an online, cross-linguistic investigation
of word recognition}. Presented at the 2022 Workshop on Infant Language
Development, San Sebastian, Spain.
\url{https://doi.org/10.6084/m9.figshare.20012729.v2}}

\leavevmode\vadjust pre{\hypertarget{ref-pVH2021}{}}%
\CSLLeftMargin{9. }%
\CSLRightInline{Von Holzen, K., \& Newman, R. (2021). \emph{The
recognition of foreign words at first exposure in early language
development: The role of phonological similarity}. Presented at the
Biennial Meeting of the Society for Research in Child Development,
Virtual Conference.
\url{https://doi.org/10.6084/m9.figshare.14381984.v1}}

\leavevmode\vadjust pre{\hypertarget{ref-pVH2018a}{}}%
\CSLLeftMargin{10. }%
\CSLRightInline{Von Holzen, K., \& Bergmann, C. (2018). \emph{A
meta-analysis of mispronunciation sensitivity in infancy}. Presented at
the XXIth Biennial International Conference on Infant Studies,
Philadelphia, PA.}

\leavevmode\vadjust pre{\hypertarget{ref-pVH2018b}{}}%
\CSLLeftMargin{11. }%
\CSLRightInline{Von Holzen, K., Van Ommen, S., White, K., \& Nazzi, T.
(2018). \emph{French-learning infants' adaptation to a novel accent: The
role of consonant/vowel asymmetry}. Presented at the XXIth Biennial
International Conference on Infant Studies, Philadelphia, PA.}

\leavevmode\vadjust pre{\hypertarget{ref-pVH2017c}{}}%
\CSLLeftMargin{12. }%
\CSLRightInline{Von Holzen, K., \& Bergmann, C. (2017). \emph{Babies
know words, even when they are mispronounced: A meta-analytic view}.
Presented at the Workshop on Infant Language Development, Bilbao, Spain.
\url{https://doi.org/10.6084/m9.figshare.5492464.v1}}

\leavevmode\vadjust pre{\hypertarget{ref-pVH2017b}{}}%
\CSLLeftMargin{13. }%
\CSLRightInline{Von Holzen, K., Rider, D., \& Nazzi, T. (2017).
\emph{Consonant and vowel processing in 5-, 8-, and 11-month-olds own
name recognition: The role of acoustic/phonetic and lexical factors}.
Presented at the Congress of the International Association for the Study
of Child Language, Lyon, France.}

\leavevmode\vadjust pre{\hypertarget{ref-pVH2017a}{}}%
\CSLLeftMargin{14. }%
\CSLRightInline{Von Holzen, K., Nishibayashi, L.-L., \& Nazzi, T.
(2017). \emph{Neural bases of phonological processing of newly segmented
word forms}. Presented at the 2017 Boston University Conference on
Language Development, Boston, MA.}

\leavevmode\vadjust pre{\hypertarget{ref-pVH2014}{}}%
\CSLLeftMargin{15. }%
\CSLRightInline{Von Holzen, K., Fennell, C. T., \& Mani, N. (2014).
\emph{The cognate facilitation effect in bilingual and monolingual
toddlers}. Presented at the XIXth Biennial International Conference on
Infant Studies, Berlin, Germany.}

\leavevmode\vadjust pre{\hypertarget{ref-pVH2013b}{}}%
\CSLLeftMargin{16. }%
\CSLRightInline{Von Holzen, K., Wolff, D., \& Mani, N. (2013).
\emph{Segmentation of IDS and ADS in 12-month-olds: An ERP study}.
Presented at the Workshop on Infant Language Development, San Sebastian,
Spain.}

\leavevmode\vadjust pre{\hypertarget{ref-pVH2013c}{}}%
\CSLLeftMargin{17. }%
\CSLRightInline{Von Holzen, K., Kremer, F., \& Mani, N. (2013).
\emph{Associating a language with a speaker: Bilingual production is
influenced by speaker language}. Presented at the International Workshop
on Bilingualism and Cognitive Control, Krakow, Poland.}

\leavevmode\vadjust pre{\hypertarget{ref-pVH2013a}{}}%
\CSLLeftMargin{18. }%
\CSLRightInline{Von Holzen, K., \& Mani, N. (2013). \emph{Native and
non-native phoneme perception in bilingual toddlers: A longitudinal
study}. Presented at the Workshop on Infant Language Development, San
Sebastian, Spain.}

\leavevmode\vadjust pre{\hypertarget{ref-pVH2011a}{}}%
\CSLLeftMargin{19. }%
\CSLRightInline{Von Holzen, K., \& Mani, N. (2011). \emph{Learning
phonemes from faces: The role of speaker identity in non-native phoneme
discrimination}. Presented at the 2011 Boston University Conference on
Language Development, Boston, MA.}

\leavevmode\vadjust pre{\hypertarget{ref-pVH2011b}{}}%
\CSLLeftMargin{20. }%
\CSLRightInline{Von Holzen, K., \& Mani, N. (2011). \emph{Word-word
relationships between languages and across development}. Presented at
the 2011 CNS Annual Meeting, San Fransicso, CA.}

\leavevmode\vadjust pre{\hypertarget{ref-pVH2010}{}}%
\CSLLeftMargin{21. }%
\CSLRightInline{Von Holzen, K., \& Mani, N. (2010). \emph{Phonological
priming across language borders: No passport required?} Presented at the
Donostia Workshop on Neurobilingualsim, Donostia-San Sebastian, Spain.}

\pagebreak

\hypertarget{research-skills}{%
\section{Research Skills}\label{research-skills}}

\hypertarget{experiment-programming-experience}{%
\subsection{Experiment Programming
Experience}\label{experiment-programming-experience}}

\begin{longtable}{@{\extracolsep{\fill}}ll}
Open Sesame/Python & \parbox[t]{0.85\textwidth}{%
\textbf{Intermediate expertise}\hfill{\footnotesize }\newline
  Designing experiments, including lexical decision (adults), visual world paradigm (with and without an eye-tracker) and preferential listening (infants); Experience advising colleagues on experimental design; Conducting tutorials with colleagues and students; Launching experiments online; Used to promote Open Science practices\par%
  \empty%
\vspace{\parsep}}\\
Presentation & \parbox[t]{0.85\textwidth}{%
\textbf{Intermediate expertise (lapsed)}\hfill{\footnotesize }\newline
  Designing experiments, including EEG experiments\par%
  \empty%
\vspace{\parsep}}\\
Eprime & \parbox[t]{0.85\textwidth}{%
\textbf{Beginner expertise}\hfill{\footnotesize }\newline
  Designing experiments, including EEG experiments\par%
  \empty%
\vspace{\parsep}}\\
\end{longtable}

\hypertarget{statisticaldata-analysis-experience}{%
\subsection{Statistical/Data Analysis
Experience}\label{statisticaldata-analysis-experience}}

\begin{longtable}{@{\extracolsep{\fill}}ll}
R & \parbox[t]{0.85\textwidth}{%
\textbf{Proficient expertise}\hfill{\footnotesize }\newline
  Advising colleagues on statistical analysis, data wrangling; Conducting tutorials; Growth growth curve modeling (lmer package); Mixed effects models (lmer package); ANOVA analyses (ez package); Producing beautiful graphics (ggplot2 package); Analyzing vocabulary data corpora (wordbankr package); Power analysis (pwr package); Creating RMarkdown files to promote collaboration and accuracy; Used to promote Open Science practices\par%
  \empty%
\vspace{\parsep}}\\
Praat & \parbox[t]{0.85\textwidth}{%
\textbf{Intermediate expertise}\hfill{\footnotesize }\newline
  Scripting to automatically analyze sounds (pitch, duration, intensity); Stimuli creation; Interfacing with CLAN output to analyze corpora; Interfacing with a forced aligner (Easy Align); Used to promote Open Science practices\par%
  \empty%
\vspace{\parsep}}\\
CHAT/CLAN & \parbox[t]{0.85\textwidth}{%
\textbf{Intermediate expertise}\hfill{\footnotesize }\newline
  Analyzing length of utterance; Time-locking transcripts to audio recordings; Interfacing with Praat\par%
  \empty%
\vspace{\parsep}}\\
EEG/ERPlab & \parbox[t]{0.85\textwidth}{%
\textbf{Intermediate expertise}\hfill{\footnotesize }\newline
  Preprocessing EEG data; Conducting tutorials; Related MATLAB scripting to automatically analyze EEG/ERP data\par%
  \empty%
\vspace{\parsep}}\\
SPSS & \parbox[t]{0.85\textwidth}{%
\textbf{Intermediate expertise (lapsed)}\hfill{\footnotesize }\newline
  ANOVA analyses\par%
  \empty%
\vspace{\parsep}}\\
\end{longtable}

\end{document}
