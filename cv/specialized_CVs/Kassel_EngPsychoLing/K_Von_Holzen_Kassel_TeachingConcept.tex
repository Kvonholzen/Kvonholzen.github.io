%% Template for a CV
%% Author: Rob J Hyndman

\documentclass[10pt,a4paper,]{article}
\usepackage[scaled=0.86]{DejaVuSansMono}
%\usepackage[sfdefault,lf,t]{carlito}

% Change color to blue
\usepackage{color,xcolor}
\definecolor{headcolor}{HTML}{990000}

\usepackage{ifxetex,ifluatex}
\usepackage{fixltx2e} % provides \textsubscript
\ifnum 0\ifxetex 1\fi\ifluatex 1\fi=0 % if pdftex
  \usepackage[T1]{fontenc}
  \usepackage[utf8]{inputenc}
\else % if luatex or xelatex
  \ifxetex
    \usepackage{mathspec}
  \else
    \usepackage{fontspec}
  \fi
  \defaultfontfeatures{Ligatures=TeX,Scale=MatchLowercase}
\fi

\usepackage[utf8]{inputenc}
\usepackage[T1]{fontenc}

% use upquote if available, for straight quotes in verbatim environments
\IfFileExists{upquote.sty}{\usepackage{upquote}}{}
% use microtype if available
\IfFileExists{microtype.sty}{%
\usepackage[]{microtype}
\UseMicrotypeSet[protrusion]{basicmath} % disable protrusion for tt fonts
}{}
\PassOptionsToPackage{hyphens}{url} % url is loaded by hyperref
\usepackage[unicode=true,hidelinks]{hyperref}
\urlstyle{same}  % don't use monospace font for urls
\usepackage{geometry}
\geometry{left=1.75cm,right=1.75cm,top=2.2cm,bottom=2cm}

\usepackage{longtable,booktabs}
% Fix footnotes in tables (requires footnote package)
\IfFileExists{footnote.sty}{\usepackage{footnote}\makesavenoteenv{long table}}{}
\IfFileExists{parskip.sty}{%
\usepackage{parskip}
}{% else
\setlength{\parindent}{0pt}
\setlength{\parskip}{6pt plus 2pt minus 1pt}
}
\setlength{\emergencystretch}{3em}  % prevent overfull lines
\providecommand{\tightlist}{%
  \setlength{\itemsep}{0pt}\setlength{\parskip}{0pt}}
\setcounter{secnumdepth}{0}

% set default figure placement to htbp
\makeatletter
\def\fps@figure{htbp}
\makeatother



\date{September 2023}


\usepackage{paralist,ragged2e,datetime}
\usepackage[hyphens]{url}
\usepackage{fancyhdr,enumitem,pifont}
\usepackage[compact,small,sf,bf]{titlesec}

\RaggedRight
\sloppy

% Header and footer
\pagestyle{fancy}
\makeatletter
\lhead{\sf\textcolor[gray]{0.4}{Teaching Concept: \@name}}
\rhead{\sf\textcolor[gray]{0.4}{\thepage}}
\cfoot{}
\def\headrule{{\color[gray]{0.4}\hrule\@height\headrulewidth\@width\headwidth \vskip-\headrulewidth}}
\makeatother

% Header box
\usepackage{tabularx}

\makeatletter
\def\name#1{\def\@name{#1}}
\def\info#1{\def\@info{#1}}
\makeatother
\newcommand{\shadebox}[3][.9]{\fcolorbox[gray]{0}{#1}{\parbox{#2}{#3}}}

\usepackage{calc}
\newlength{\headerboxwidth}
\setlength{\headerboxwidth}{\textwidth}
%\addtolength{\headerboxwidth}{0.2cm}
\makeatletter
\def\maketitle{
\thispagestyle{plain}
\vspace*{-1.4cm}
\shadebox[0.9]{\headerboxwidth}{\sf\color{headcolor}\hfil
\hbox to 0.98\textwidth{\begin{tabular}{l}
\\[-0.3cm]
\LARGE\textbf{\@name}
\\[0.1cm]\large Lecturer and Researcher\\[0.6cm]
\normalsize\textbf{Teaching Concept}\\
\normalsize September 2023
\end{tabular}
\hfill\hbox{\fontsize{9}{12}\sf
\begin{tabular}{@{}rl@{}}
\@info
\end{tabular}}}\hfil
}
\vspace*{0.2cm}}
\makeatother

% Section headings
\titlelabel{}
\titlespacing{\section}{0pt}{1.5ex}{0.5ex}
\titleformat*{\section}{\color{headcolor}\large\sf\bfseries}
\titleformat*{\subsection}{\color{headcolor}\sf\bfseries}
\titlespacing{\subsection}{0pt}{1ex}{0.5ex}

% Miscellaneous dimensions
\setlength{\parskip}{0ex}
\setlength{\parindent}{0em}
\setlength{\headheight}{15pt}
\setlength{\tabcolsep}{0.15cm}
\clubpenalty = 10000
\widowpenalty = 10000
\setlist{itemsep=1pt}
\setdescription{labelwidth=1.2cm,leftmargin=1.5cm,labelindent=1.5cm,font=\rm}

% Make nicer bullets
\renewcommand{\labelitemi}{\ding{228}}

\usepackage{booktabs,fontawesome5}
%\usepackage[t1,scale=0.86]{sourcecodepro}

\name{Katie Von Holzen}
\def\imagetop#1{\vtop{\null\hbox{#1}}}
\info{%
\raisebox{-0.05cm}{\imagetop{\faIcon{map-marker-alt}}} &  \imagetop{\begin{tabular}{@{}l@{}}Hagenring
22, 38106 Braunschweig, Germany\end{tabular}}\\ %
\faIcon{home} & \href{http://Kvonholzen.github.io}{Kvonholzen.github.io}\\% %
%
%
\faIcon{envelope} & \href{mailto:katie.m.vonholzen@gmail.com}{\nolinkurl{katie.m.vonholzen@gmail.com}}\\%
\faIcon{twitter} & \href{https://twitter.com/KatieVonHolzen}{@KatieVonHolzen}\\%
\faIcon{github} & \href{https://github.com/Kvonholzen}{Kvonholzen}\\%
%
%
%
%
}


%\usepackage{inconsolata}


\setlength\LTleft{0pt}
\setlength\LTright{0pt}

% Pandoc CSL macros
\newlength{\cslhangindent}
\setlength{\cslhangindent}{1.5em}
\newlength{\csllabelwidth}
\setlength{\csllabelwidth}{3em}
\newenvironment{CSLReferences}[3] % #1 hanging-ident, #2 entry spacing
 {% don't indent paragraphs
  \setlength{\parindent}{0pt}
  % turn on hanging indent if param 1 is 1
  \ifodd #1 \everypar{\setlength{\hangindent}{\cslhangindent}}\ignorespaces\fi
  % set entry spacing
  \ifnum #2 > 0
  \setlength{\parskip}{#2\baselineskip}
  \fi
 }%
 {}
\usepackage{calc}
\newcommand{\CSLBlock}[1]{#1\hfill\break}
\newcommand{\CSLLeftMargin}[1]{\parbox[t]{\csllabelwidth}{\hfill #1~}}
\newcommand{\CSLRightInline}[1]{\parbox[t]{\linewidth - \cslhangindent - \csllabelwidth}{#1}\vspace{0.8ex}}
\newcommand{\CSLIndent}[1]{\hspace{\cslhangindent}#1}


\def\endfirstpage{\newpage}

\begin{document}
\maketitle


As a teacher, I understand my role to be that of a guide. I know the
goal (learning outcome) and I know at least one way to get there (my own
way), but it is my job to accompany the students as they find their
individual ways through their learning process. In this regard, I
especially focus on the student as an individual and understand that
each student comes to my class with their own needs, wants, skills, and
interests. I therefore try to create my courses such that all learners
can be successful, integrating sections where I give information,
sections that encourage discussion or interaction amongst the students,
and sections where students need to apply what they've learned to a
task. I have come to this teaching approach through my years of teaching
in the United States and Germany as well as through the NRW Teaching
Certificate Program ``Professionelle Lehrkompetenz für die Hochschule''.
\linebreak

In my courses, I encourage students to be critical thinkers. For
example, I design my advanced seminar courses to be led by students in a
structured discussion. For each class, students read a scientific
article or book chapter and prepare a brief summary of the topic, their
remaining questions, and information they found surprising or
interesting. To ensure students come to class prepared, a few students
are randomly chosen to present their prepared materials and the entire
class is invited to add their summaries and questions. I then guide the
class discussion around the points that the students presented. This
teaches students how to critique and think critically through dialogue
with their peers, while demonstrating a deep understanding of the
material, which will be an essential skill in their scholarly and
professional lives. \linebreak

I place great importance on scientific literacy in my teaching. I use
the QALMRI method (Brosowsky \& Parshina, 2017) to teach students how to
identify the questions being asked in a scientific article, as well as
how those questions are answered and the consequences of those answers.
I've used this method in both a first semester class on brain and
behavioral research, investigating topics such as the ``bilingual
advantage'' from both the perspective of popular media and from primary
scientific articles, as well as more traditional master's seminars on
language processing and bilingualism research. Reading and understanding
scientific articles can be daunting for students, but I've found that
when I guide my students using a scaffold for approaching and processing
scientific articles, our discussions of those articles reflect a deep
level of learning. \linebreak

Through the NRW Teaching Certificate Program I've learned several
strategies for effective teaching. One central insight is how to design
assignments for students that support their ability to apply their
knowledge to real-world situations. For example, in my seminar
``Introduction to Linguistics'' at the TU Dortmund, I introduced
students to the basic concepts of Linguistics through the eyes of a
young child learning their first language. In their assignments,
students tested their own hypothesis about how vocabulary develops in
children using the Wordbank database
(\url{http://wordbank.stanford.edu/}) or analyzed a child's stage in the
acquisition of morpho-syntax using the CHILDES database
(\url{https://childes.talkbank.org/}). \linebreak

I look forward to applying my teaching approach at the University of
Kassel by taking over existing classes or developing new classes within
the Institute for English and American Studies. The BA, MA, and teacher
training specializations in English and American Studies offer classes
that I am qualified for and excited to teach, ranging from content
courses (e.g.~Introduction to Linguistics) to methods and statistics
(e.g.~Processing Sentences: Experimental Methods in Linguistics). I
would also be interested in developing new courses, such as
\emph{Exploring Psycholinguistics through Open Science}, as well as
adapting my existing seminars, such as \emph{Listening in the L1 and L2,
Word Segmentation: Findings from Psycho- and Neurolinguistics}, or
\emph{Two Languages, One Mind: Bilingualism and its Consequences}. I
would also be interested in contributing to and supporting the
organization of the lecture series ``Linguistik U Kassel''.

\end{document}
